\documentclass{article}
\usepackage{
    amsthm,
    amsmath,
    amssymb,
    mathtools
}

\theoremstyle{definition}
\newtheorem{definition}{Definition}[subsection]
\newtheorem{theorem}{Theorem}[subsection]
\newtheorem{example}{Example}[section]

\begin{document}
\section{Signal \& Systems (Fundamental)}
\subsection{Signal Definition}
\begin{definition}[Signal]
    Any "physical" quantity that varies with time or space (or other independetn variables).
\end{definition}
\begin{example}[Ambulance Siren:]
    \begin{equation}
        s(t) = (1+t)sin(2\pi [1000t+10t^2+300sin(2\pi t /2)]) 
    \end{equation}
    \begin{itemize}
        \item $ (1+t) $: amplitude term represents incresing loudness as ambulance approaches
        \item $ 1000t $: represents 1kHz siren oscillatione
        \item $ 10t^2 $: increasing pitch due to the \emph{Doppler effect} as the ambulance approaches.
        \item $ 300sim(2\pi 2t) $: the eeh-ooh-eeh-ohh periodic variation in pitch.
    \end{itemize} 
\end{example}

\begin{definition}[Systems]
    A physical "devicec" that performs an operation on a signal.
\end{definition}
\begin{definition}[Signal Processing]
    Take some input signals and produce some related output signals.
\end{definition}
\begin{example}[audio amplifier]
    \begin{equation}
        S_{out}(t) = aS_{in}(t)
    \end{equation}
\end{example}
Emphasize on continuous-time of analog signals.
\section{Classification of Signals}
\subsection{Dimensionality}
\begin{itemize}
    \item By the domain of the function, i.e.\ how many arguments a function has.
    \item By the dimension of the range of the function, i.e.\ the space of values the funciton can take.
\end{itemize}
\subsection{Time characteristics}
\begin{definition}[Continuous-time Signal]
    A function defined for all times $ t \in (-\infty,\infty) $, or at least some interval $ (a,b) $.
\end{definition}
Classify signals by time characteristics
\begin{itemize}
    \item Continuous-11:10 signals or analog signals
    \item Discrete-time signals
\end{itemize}
























\end{document}

