\documentclass{article}
\usepackage{
    amsthm,
    amsmath,
    amssymb,
    mathtools
}

\theoremstyle{definition}
\newtheorem{definition}{Definition}[subsection]
\newtheorem{theorem}{Theorem}[subsection]
\newtheorem{example}{Example}[section]

\begin{document}
\tableofcontents

\section{Signal \& Systems (Fundamental)}
\subsection{Signal Definition}
\begin{definition}[Signal]
    Any "physical" quantity that varies with time or space (or other independetn variables).
\end{definition}
\begin{example}[Ambulance Siren:]
    \begin{equation}
        s(t) = (1+t)sin(2\pi [1000t+10t^2+300sin(2\pi t /2)]) 
    \end{equation}
    \begin{itemize}
        \item $ (1+t) $: amplitude term represents incresing loudness as ambulance approaches
        \item $ 1000t $: represents 1kHz siren oscillatione
        \item $ 10t^2 $: increasing pitch due to the \emph{Doppler effect} as the ambulance approaches.
        \item $ 300sim(2\pi 2t) $: the eeh-ooh-eeh-ohh periodic variation in pitch.
    \end{itemize} 
\end{example}

\begin{definition}[Systems]
    A physical "devicec" that performs an operation on a signal.
\end{definition}
\begin{definition}[Signal Processing]
    Take some input signals and produce some related output signals.
\end{definition}
\begin{example}[audio amplifier]
    \begin{equation}
        S_{out}(t) = aS_{in}(t)
    \end{equation}
\end{example}
Emphasize on continuous-time of analog signals.
\subsection{Classification of Signals}
\subsubsection{Dimensionality}
\begin{itemize}
    \item By the domain of the function, i.e.\ how many arguments a function has.
    \item By the dimension of the range of the function, i.e.\ the space of values the funciton can take.
\end{itemize}
\subsubsection{Time characteristics}
\begin{definition}[Continuous-time Signal]
    A function defined for all times $ t \in (-\infty,\infty) $, or at least some interval $ (a,b) $.
\end{definition}
Classify signals by time characteristics
\begin{itemize}
    \item Continuous-11:10 signals or analog signals
    \item Discrete-time signals
\end{itemize}
\subsubsection{Value Characteristics}
\begin{definition}[continuous-valued signal or continuous-amplitude signal]
    Can take any value in some continuous interval.
\end{definition}
\begin{definition}[discrete-valued signal or discrete-amplitude signal]
    Only takes values from a discrete set of possible values.
\end{definition}
\subsubsection{Determinstic vs Random Signals}
\begin{definition}[Determinstic signals]
    Can be described by an explicit mathematical representation.
\end{definition}
\begin{definition}[Random signals]
    Evolve over time in an unpredictable manner.
\end{definition}

\subsection{Transformation of CT Signals}
\subsubsection{Transformations}
\begin{itemize}
    \item Time transformations
        \begin{itemize}
            \item Folding/reflecting/time-reversal
                \begin{equation}
                    y(t)=x(-t)
                \end{equation}
            \item Time-scaling
                \begin{equation}
                    y(t)=x(at)
                \end{equation}
            \item Time-shifting
                \begin{equation}
                    y(t)=x(t-t_0)
                \end{equation}
            \item General time transformations\\
                Involves all three of the above time transformations.
                \begin{equation}
                    y(t) = x(at-b)=x(\frac{t-t_0}{w})
                \end{equation}
                where $ t_0 = b/a $, $ w=1/a $
        \end{itemize}
    \item Amplitude transformations
        \begin{itemize}
            \item reverse\\
                \begin{equation}
                    y(t)=-x(t)
                \end{equation}
            \item scaling
                \begin{equation}
                    y(t) = ax(t)
                \end{equation}
            \item shifting
                \begin{equation}
                    y(t)=x(t)+b
                \end{equation}
        \end{itemize}
    \item Differentiator
        \begin{equation}
            y(t)=\frac{d}{dt} x(t)
        \end{equation}
        \begin{example}
            \begin{equation}    
                y(t)=-RC \frac{d}{dt}x(t)
            \end{equation}
        \end{example}
    \item Integrator
        \begin{equation}
            y(t)=\int_{-\infty}^t x(\tau)d\tau
        \end{equation}
        \begin{example}
            \begin{equation}    
                y(t)=-\frac{1}{RC} \int_{-\infty}^t x(\tau)d\tau
            \end{equation}
        \end{example}
    \item Operation with two signals\\
        Sum or product at any point
\end{itemize}

\subsection{Signal Characteristic}
\subsubsection{Periodic signals}
\begin{equation}
    x(t+T)=x(t)\forall t
\end{equation}
if no $ T $ exists, called aperiodic.
\begin{definition}[Fundamental Period]
    Smallest $ T_0 $ of $ T $.
\end{definition}
\begin{theorem}
    With period $ T>0 $,
    \begin{equation}
        x(t+nT) = x(t)
    \end{equation}
\end{theorem}

Sum of two periodic signals\\
Suppose a value $ T>0 $ satisfies $ T=n_1T_1 $ and $ T=n_2T_2 $\\
then, x(t) is periodic with period T.
\begin{theorem}
    A sum of two periodic signals is period iff the ratio od their periods is rational.
\end{theorem}
\subsubsection{Even and Odd Symmetry}
\begin{definition}[Even Symmetry]
    iff $ x(-t) = x(t)\forall t $
\end{definition}
\begin{definition}[Odd Symmetry]
    iff $ x(-t)=x(t)\forall t $
\end{definition}
Even and Odd Components\\
We can decompose any signal into even and odd components:
\begin{equation}
    x(t) = x_e(t)+x_o(t)
\end{equation}
\begin{equation}
    x_e(t)=\frac{1}{2}[x(t)+x(-t)],\quad x_o(t) = \frac{1}{2}[x(t)-x(-t)]
\end{equation}
\subsubsection{Average value and energy}
\begin{definition}[Average Value]
    \begin{equation}
        A = \lim_{T \to \infty} \frac{1}{2T} \int_{-T}^{T}x(t)dt
    \end{equation}
\end{definition}
\begin{definition}[Energy]
    \begin{equation}
        E = \int_{-\infty}^{\infty} |x(t)|^2dt
    \end{equation}
\end{definition}
\begin{definition}[Average Power] 
    \begin{equation}
        P =\lim_{T \to \infty} \frac{1}{2T} \int_{-T}^{T} |x(t)|^2dt
    \end{equation}
\end{definition}
\begin{definition}[Energy Signal]
    If $ E $ is finite, then $ x(t) $ is called energy signal and $ P=0 $.
\end{definition}
\begin{definition}[Power Signal]
    If $ E $ is infinite and $ P $ is finite and nonzero, $ x(t) $ is called power signal.
\end{definition}
\subsection{Exponential signals}
To be done
\subsection{Continuous0-time system}
\begin{definition}[Continuous-time(CT) System]
    a device that transforms input CT signal into another output CT signal.
\end{definition}
\begin{equation}
    y(\cdot) = \mathcal{T}(x(\cdot))
\end{equation}
\begin{definition}[Input-output Relationship]
    Precisely defines how the output signal is related to the input signal.
\end{definition}
\begin{itemize}
    \item Series connection
        \begin{equation}
            x(t)\rightarrow \boxed{\mathcal{T}_1}\rightarrow \boxed{\mathcal{T}_2} \rightarrow y(t)
        \end{equation}
        \begin{equation}
            y(t)=\mathcal{T}_2[\mathcal{T}_1[x(t)]]
        \end{equation}
    \item Parallel connection
        \begin{equation}
            y(t)=\mathcal{T}_1[x(t)]+\mathcal{T}_2[x(t)]
        \end{equation}
\end{itemize}
\subsubsection{Classification of CT Systems}
\begin{itemize}
    \item Amplitude properties
        \begin{itemize}
            \item A-1 linearity
                \begin{equation}
                    \mathcal{T}[a_1x_1(t)+a_2x_2(t)] = a_1\mathcal{T}[x_1(t)]+a_2 \mathcal{T}[x_2(t)]
                \end{equation}
                \textbf{Property}\\
                \begin{equation}
                    \mathcal{T}[ax(t)] = a\mathcal{T}[x(t)]
                \end{equation}
                superpostion property
                \begin{equation}
                    \mathcal{T}[\sum_{k=1}^{K} \mathcal{T}[x_k(t)]]
                \end{equation}
                \begin{equation}
                    \mathcal{T}[\int x(t;v)dv]=\int \mathcal{T}[x(t;v)]dv
                \end{equation}
            \item A-2 stability\\
                Satisfy BIBO
               \begin{definition}[Bounded-input Bounded-output (BIBO) stable]
                   Every bounded input produces a bounded output
               \end{definition}
               \textbf{Triangle Inquality}\\
               \begin{equation}
                   |\sum_n a_n| \leq \sum_n |a_n|
               \end{equation}
            \item invertibility
                \begin{definition}[invertible]
                    each output signal is the response to only one input signal.
                \end{definition}
                \textbf{Property}\\
                \begin{equation}
                    \mathcal{T}^{-1}[\mathcal{T}[x(t)]]=x(t)
                \end{equation}
        \end{itemize}
    \item Time properties
        \begin{itemize}
            \item T-1 causality
                \begin{definition}[Casual System]
                    The output $ y(t) $ at time $ t $ only depends on the present and (possibly) past inputs, not on future inputs.
                \end{definition}
                Noncasual systems arise often when t is other variables than time, such as space.
            \item T-2 memory
                \begin{definition}[Static system or Memoryless System]
                    The output $ y(t) $ at time $ t $ only depends on the current input $ x(t) $.\\
                    Otherwise is a dynamic system.
                \end{definition}                
            \item T-3 time-invariance
                Systems whose input-output behavior does not change with time
                \begin{definition}[Time Invariant]
                    \begin{equation}
                        x(t)\mathop{\rightarrow}\limits^{\mathcal{T}}y(t) \quad implies\; that \quad x(t-t_0)\mathop{\rightarrow}\limits^{\mathcal{T}}y(t-t_0)
                    \end{equation}
                \end{definition}
        \end{itemize}
\end{itemize}






















\end{document}

