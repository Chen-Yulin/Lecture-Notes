\documentclass{article}
\usepackage{
    amsthm,
    amsmath,
    amssymb,
    mathtools
}

\theoremstyle{definition}
\newtheorem{definition}{Definition}[subsection]
\newtheorem{theorem}{Theorem}[subsection]

\begin{document}
\section{Probability Theory}
\subsection{Elemantary Probability}
\begin{definition}[Cardano's Principle]
    A be a random outcome of an experiment that may proceed in various ways.
    Assume each of these ways is \textbf{equally likely}, then, probability $P[ A]$ of outcome A is
    \begin{equation}
        P[A] = \frac{number\; of\; ways\; leading\; to\; outcome\; A}{number\; of\; ways\; the\; experiment\; can\; be\; proceeded}
    \end{equation}
\end{definition}
\subsubsection{Basic principles of Counting}
\begin{itemize}
    \item Permutation:
        \begin{equation}
            A_n^k = \frac{n!}{(n-k)!}
        \end{equation}
    \item Combination:
        \begin{equation}
            \tbinom{n}{k}=\frac{n!}{(n-k)!k!}
        \end{equation}
    \item Permutation of k \textbf{Indisguishable} Objects:
        \begin{equation}
            \frac{n!}{n_1!n_2!n_3!\dotsm n_k!}=\tbinom{n}{n_1}\cdot \tbinom{n-n_1}{n_2} \dotsm \tbinom{n-(n_1+n_2 + \dotsm + n_{k-1})}{n_k}
        \end{equation}
        \emph{Remark}:\\ 
        In permutation of $k$ indistinguishable objects, the elements has noorder within a group but are differnent from each other; the groups either has order or are different from each other.\\
        \emph{Example}:\\
        Consider 10 balls, 5 red, 3 green and 2 blue. How many ways can they be arranged on a line?
        \begin{equation}
            \frac{10!}{5! \cdot 3! \cdot 2!} = 2520
        \end{equation}
\end{itemize}
\subsubsection{Sample Points, Sample Space and $\sigma$-Field }
\begin{definition}[Sample Points]
    Mathematical objects are called sample points.
\end{definition}
\begin{definition}[Sample Space]
    The sample space $S$ is large enough to accommodate all the sample points.
\end{definition}
\begin{definition}[Event]
    An outcome in the sense of Cardano's principle is interpreted as a subset $A$ of a sample space S abd called an event.
\end{definition}
\begin{definition}[Mutually exclusive]
    Two events $A_1$, $A_2$ are called mutual exclusive if $A_1\cap A_2=\emptyset$
\end{definition}














\end{document}

