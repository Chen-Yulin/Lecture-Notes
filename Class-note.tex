\documentclass{article}
\usepackage{
    CJK,
    amsthm,
    amsmath,
    amssymb,
    mathtools
}

\theoremstyle{definition}
\newtheorem{definition}{Definition}[subsection]
\newtheorem{theorem}{Theorem}

\begin{document}

\section{Basic Set Theory}

Paul R.~Halmos, Naive Set Theory.
\begin{definition}[Set]
    A set is an unordered colleciton of distinct objects, called \emph{elements} or \emph{menbers} of the set. A set is said to contain its elements. We write
    \begin{itemize}
        \item $a \in A$ if $a$ is an element of the set $A$.
    \end{itemize}
\end{definition}
\subsection{Set Operation}
\begin{align*}
    U & = \{x \mid x \in X, x \in \mathbb{C} \} \\
      & = U_{x \in C}C
\end{align*}

\subsubsection{Set difference}
The set difference of A and B ,denoted by $A-B$, or $A \backslash B$
\subsubsection{Symmetric Difference}
\begin{equation}
    A \Delta B = (A-B)\cup(B-A)
\end{equation}

\subsubsection{Power Set}
The power set of A is the set of all subsets of A, denoted by $\mathcal{P}(A)$ or $2^A$

\subsubsection{Set Algebras}
De Morgan's Laws
\begin{itemize}
    \item $C - (A \cup B) = (C-A) \cap (C-B)$
    \item $C - (A \cap B) = (C-A) \cup (C-B) $
\end{itemize}

\subsubsection{Cartesian Product}
\begin{definition}[Cartesian]
    The Cartesian product of sets A and B is the set of ordered pairs, such that
    \begin{equation}
        A \times B = \{ (a,b) \mid a \in A, b \in B \}
    \end{equation}
\end{definition}
\begin{definition}[Cartesian]
    By Kuratowski\\
    An ordered pair $(a,b)$ is given by
    \begin{equation}
        (a,b):=\{ \{a\}, \{a,b\} \}
    \end{equation}
\end{definition}
\begin{theorem}[Cartesian]
    If $a \in C$ and $b \in C$, then $(a,b) = \mathcal{P}(\mathcal{P}(C))$.\\
    If $a \in A, b \in B$, then take $C = A \cup B$
\end{theorem}
\begin{theorem}[Cartesian]
    $(x,y)=(a,b)$ \it{iff} $x=a$ and $y=b$
\end{theorem}
\boldmath{Proof}
\begin{itemize}
    \item $\Leftarrow$: Trival
    \item $\Rightarrow$: By definition, we need to show that:
       \begin{equation}
           \underbrace{\{\{x\},\{x,y\}\}}_{{}U} = \underbrace{\{\{a\},\{a,b\}\}}_{{}V} \Rightarrow x=a\; and\; y=b
       \end{equation} 
       \begin{itemize}
           \item If $x \neq y$, then $\lvert U \rvert  = \lvert V\rvert = 2$, by matching size, we have $\{x\}=\{a\} \; \{y\}=\{b\}$
           \item If $x=y$, similarly $x=y=a=b$
       \end{itemize}
\end{itemize}
Note: a definitio of ordered pairs is rational as long as it can indicate order.
\subsubsection{Associative Set Operations}
Let $A_1,A_1,\dotsm,A_n$ be sets, then
\begin{itemize}
    \item $A_1 \cap A_2 \cap \dotsm \cap A_n = \mathop{\cap}\limits_{i=1}^{n}A_i$
    \item for $\cup,\times,\Delta$ are the same.
\end{itemize}
\subsection{Simple Graphs}
k-element subsets\\
Let X be a finit set. For a positive integer k, let $\tbinom{X}{k}$ denote the set of all k-element subsets. Note that $\lvert \tbinom{X}{k}\rvert = \tbinom{\lvert X\rvert}{k}$
\begin{definition}[Graph]
    A finit simple \it{graph} $G$ is a pair $(V,E)$ where V is a non-empty finit set and E is a set of 2-element subsets of V, i.e., $E \in \tbinom{V}{2}$
    Elements of V called vertics, also denoted as $V(G)$.
    Elements of E called edges,also denoed as $E(G)$.
\end{definition}

\section{Logic}
\subsection{CNF and DNF}
\begin{itemize}
    \item Conjunctive Normal Form (CNF):\\
        a junction of one or more clauses, where a clause is a disjunction of literals\\
        like \textbf{roduct of sums} or \textbf{AND of ORs}
    \item Disjunction Normal Forn (DNF):\\
        like \textbf{sum of products} or \textbf{OR of ANDs}
\end{itemize}
\subsection{Conditional Statements}
\begin{table}[h]
    \centering
    \caption{Conditional truth table}

    \begin{tabular}{|c|c|c|}
        \hline  $p$   &   $q$   &   $p \rightarrow q$   \\
        \hline  0   &   0   &   1   \\
        \hline  0   &   1   &   1   \\
        \hline  1   &   0   &   0   \\
        \hline  1   &   1   &   1   \\
        \hline
    \end{tabular}
\end{table}
\begin{itemize}
    \item $p$: hyposhesis
    \item $q$: thesis/conclusion
\end{itemize}
\textbf{Equivalent forms}
\begin{itemize}
    \item if p, then q
    \item q is a sufficient condition for q
    \item q is necessary for p
\end{itemize}
\textbf{remark}
\begin{itemize}
    \item either $p$ is false
    \item or $q$ is true
    \item i.e. $\neg p \lor q$
    \item same as $\neg q \rightarrow \neg p$
\end{itemize}
\subsection{Tautology and Contradiction}
\begin{itemize}
    \item Tautology: All cases evaluates to 1.
    \item Contradiction: All cases evaluates to 0.
\end{itemize}
\subsubsection{Tautological Equivalence}
\begin{itemize}
    \item Absorption
        \begin{align*}
            &   p \land (p \lor q) \Leftrightarrow p    \\
            &   p \lor (p \land q) \Leftrightarrow p
        \end{align*}
    \item Cases
        \begin{align*}
            & (p\rightarrow q)\land(p\rightarrow r) \Leftrightarrow p\rightarrow (q\land r)\\
            & (p\rightarrow q)\lor(p\rightarrow r) \Leftrightarrow p\rightarrow (q\lor r)\\
            & (p\rightarrow r)\land(p\rightarrow r) \Leftrightarrow (p\lor q)\rightarrow r\\
            & (p\rightarrow r)\lor(p\rightarrow r) \Leftrightarrow (p\land q)\rightarrow r
        \end{align*}
    \item Added premise
        \begin{align*}
            (p\land q)\rightarrow r &\Leftrightarrow p\rightarrow(q\rightarrow r)\\
                                    &\Leftrightarrow q\rightarrow (p\rightarrow r)
        \end{align*}
\end{itemize}
\subsubsection{CNF and DNF}
\begin{theorem}[Disjunctive Normal Form]
    For any proposition $ \varphi $, there is a proposition $ \varphi_{dnf} $ over same Boolean variables and in DNF such that $ \varphi \Leftrightarrow \varphi_{dnf} $
\end{theorem}
Example:
\begin{align*}
    & \varphi=p\rightarrow q \qquad &\varphi_{dnf}=(\lnot p)\lor (q)\\
    & \varphi = p \leftrightarrow q &\varphi_{dnf} = (p\land q)\lor (\lnot q \land \lnot p)\\
    & \varphi = p \oplus q &\varphi_{dnf} = (\lnot p \land q)\lor (p\land \lnot q)
\end{align*}
Just like sum of product.
\begin{theorem}[Conjunctive Normal Form]
    For any proposition $ \varphi $, there is a proposition $ \varphi_{cnf} $ over same Boolean variables and in DNF such that $ \varphi \Leftrightarrow \varphi_{cnf} $
\end{theorem}
Example:
\begin{align*}
    & \varphi=p\rightarrow q \qquad &\varphi_{dnf}=(\lnot p\lor q)\\
    & \varphi = p \leftrightarrow q &\varphi_{dnf} = (\lnot p\lor q)\land (\lnot q \lor p)\\
    & \varphi = p \oplus q &\varphi_{dnf} = (p \lor q)\land (\lnot p\lor \lnot q)
\end{align*}
Just like product of sum.











\end{document}
